\documentclass{article}
\usepackage{amsmath,amssymb}
\usepackage{multicol}
\usepackage{geometry}
\geometry{a4paper, landscape, margin=1cm}

\title{信号处理核心公式汇总}
\author{}
\date{}

\begin{document}
\maketitle
\begin{multicols*}{5}

\section{傅里叶级数}
连续周期信号:\(f(t)=f(t+T)\),基波角频率 \(\omega_{0}=\frac{2 \pi}{T}\)

\subsection{数学形式}
- 三角形式:
  \[
  f(t)=a_{0}+\sum_{n=1}^{\infty}\left[a_{n} \cos(n \omega_{0} t)+b_{n} \sin(n \omega_{0} t)\right]
  \]
- 指数形式:
  \[
  f(t)=\sum_{n=-\infty}^{\infty} c_{n} e^{j n \omega_{0} t}
  \]
  其中傅里叶系数:
  \[
  c_{n}=\frac{1}{T} \int_{T} f(t) e^{-j n \omega_{0} t} dt
  \]

\subsection{核心关系与性质}
- 系数关系(\(f(t)\) 为实信号):
  \[
  c_{0}=a_{0}, \quad c_{n}=\frac{a_{n}-j b_{n}}{2}, \quad c_{-n}=c_{n}^{*}
  \]
- 线性性:\(a f(t)+b g(t) \leftrightarrow a c_{n}+b d_{n}\)
- 时移性:\(f(t-t_{0}) \leftrightarrow c_{n} e^{-j n \omega_{0} t_{0}}\)
- 共轭对称性:\(f(t)\) 实 \(\Rightarrow c_{-n}=c_{n}^{*}\)
- 帕塞瓦尔定理:
  \[
  \frac{1}{T} \int_{T}|f(t)|^{2} dt=\sum_{n=-\infty}^{\infty}\left|c_{n}\right|^{2}
  \]

\section{傅里叶变换}
\subsection{定义}
- 正变换:
  \[
  F(\omega)=\mathcal{F}[f(t)]=\int_{-\infty}^{\infty} f(t) e^{-j \omega t} dt
  \]
- 逆变换:
  \[
  f(t)=\mathcal{F}^{-1}[F(\omega)]=\frac{1}{2 \pi} \int_{-\infty}^{\infty} F(\omega) e^{j \omega t} d \omega
  \]

\subsection{主要性质}
1. 线性性:\(\mathcal{F}[a f(t)+b g(t)]=a F(\omega)+b G(\omega)\)
2. 时移性:\(\mathcal{F}[f(t-t_{0})]=F(\omega) e^{-j \omega t_{0}}\)
3. 频移性:\(\mathcal{F}[f(t) e^{j \omega_{0} t}]=F(\omega-\omega_{0})\)
4. 尺度变换:\(\mathcal{F}[f(a t)]=\frac{1}{|a|} F\left(\frac{\omega}{a}\right)\)
5. 对称性(对偶性):\(\mathcal{F}[F(t)]=2 \pi f(-\omega)\)
6. 时域微分:\(\mathcal{F}\left[\frac{d^{n} f(t)}{d t^{n}}\right]=(j\omega)^{n}F(\omega)\),条件:\(f(\pm\infty)=0\)
7. 频域微分:\(\mathcal{F}\left[t^{n} f(t)\right]=j^{n} \frac{d^{n} F(\omega)}{d \omega^{n}}\)
8. 积分性:\(\mathcal{F}\left[\int_{-\infty}^{t} f(\tau) d \tau\right]=\frac{F(\omega)}{j \omega}+\pi F(0) \delta(\omega)\)
9. 时域卷积:\(\mathcal{F}[f(t) * g(t)]=F(\omega) G(\omega)\)
10. 频域卷积:\(\mathcal{F}[f(t) g(t)]=\frac{1}{2 \pi} F(\omega) * G(\omega)\)
11. 帕塞瓦尔定理:
    \[
    \int_{-\infty}^{\infty}|f(t)|^{2} dt=\frac{1}{2 \pi} \int_{-\infty}^{\infty}|F(\omega)|^{2} d \omega
    \]
12. 共轭对称性:若 \(f(t)\) 实,则 \(F(-\omega)=F^{*}(\omega)\)

\subsection{奇偶虚实性}
- \(f(t)\) 实偶 \(\Rightarrow F(\omega)\) 实偶
- \(f(t)\) 实奇 \(\Rightarrow F(\omega)\) 纯虚奇
- \(f(t)\) 虚偶 \(\Rightarrow F(\omega)\) 虚偶
- \(f(t)\) 虚奇 \(\Rightarrow F(\omega)\) 实奇

\subsection{奇偶分解}
- 偶部:\(f_{e}(t)=\frac{f(t)+f(-t)}{2} \Leftrightarrow F_{e}(\omega)=\text{Re}\{F(\omega)\}\)
- 奇部:\(f_{o}(t)=\frac{f(t)-f(-t)}{2} \Leftrightarrow F_{o}(\omega)=j \text{Im}\{F(\omega)\}\)
- 关系:\(f(t)=f_{e}(t)+f_{o}(t)\),\(F(\omega)=F_{e}(\omega)+F_{o}(\omega)\)

\subsection{能量与功率}
- 能量信号:\(E=\int_{-\infty}^{\infty}|f(t)|^{2} dt=\frac{1}{2 \pi} \int_{-\infty}^{\infty}|F(\omega)|^{2} d \omega\)
- 功率信号(周期):\(P=\frac{1}{T} \int_{T}|f(t)|^{2} dt=\sum_{n=-\infty}^{\infty}|c_{n}|^{2}\)(\(c_n\) 为傅里叶系数)
- 平均功率:\(P=\lim _{T \to \infty} \frac{1}{T} \int_{-T/2}^{T/2}|f(t)|^{2} dt\)

\subsection{常见傅里叶变换对}
\[
\begin{aligned}
\delta(t) &\leftrightarrow 1 \\
1 &\leftrightarrow 2 \pi \delta(\omega) \\
e^{j \omega_{0} t} &\leftrightarrow 2 \pi \delta(\omega-\omega_{0}) \\
\cos(\omega_{0} t) &\leftrightarrow \pi\left[\delta(\omega-\omega_{0})+\delta(\omega+\omega_{0})\right] \\
\sin(\omega_{0} t) &\leftrightarrow j \pi\left[\delta(\omega+\omega_{0})-\delta(\omega-\omega_{0})\right] \\
e^{-a t} u(t) &\leftrightarrow \frac{1}{a+j \omega} \quad (a>0) \\
t e^{-a t} u(t) &\leftrightarrow \frac{1}{(a+j \omega)^{2}} \quad (a>0) \\
e^{-a|t|} &\leftrightarrow \frac{2 a}{a^{2}+\omega^{2}} \quad (a>0) \\
u(t) &\leftrightarrow \pi \delta(\omega)+\frac{1}{j \omega} \\
\text{rect}\left(\frac{t}{\tau}\right) &\leftrightarrow \tau \cdot \frac{\sin(\omega \tau/2)}{\omega \tau/2} \\
\text{sinc}(t)=\frac{\sin(\pi t)}{\pi t} &\leftrightarrow \text{rect}\left(\frac{\omega}{2 \pi}\right) \\
\text{tri}\left(\frac{t}{\tau}\right) &\leftrightarrow \tau \cdot \frac{\sin^{2}(\omega \tau/2)}{(\omega \tau/2)^{2}} \\
\sum_{n=-\infty}^{\infty} \delta(t-n T) &\leftrightarrow \frac{2 \pi}{T} \sum_{k=-\infty}^{\infty} \delta(\omega-k \omega_{0})
\end{aligned}
\]

\subsection{周期信号的傅里叶变换}
- 方法一(傅里叶级数):若 \(f(t)=\sum_{n=-\infty}^{\infty} c_{n} e^{j n \omega_{0} t}\),则
  \[
  F(\omega)=2 \pi \sum_{n=-\infty}^{\infty} c_{n} \delta(\omega-n \omega_{0})
  \]
- 方法二(单周期信号法):若 \(f(t)=\sum_{k=-\infty}^{\infty} f_{0}(t-k T)\),则
  \[
  F(\omega)=F_{0}(\omega) \cdot \frac{2 \pi}{T} \sum_{n=-\infty}^{\infty} \delta(\omega-n \omega_{0})
  \]
  其中 \(c_{n}=\frac{1}{T} F_{0}(n \omega_{0})\)

\section{离散时间傅里叶变换(DTFT)}
\subsection{定义}
- 正变换:
  \[
  X(e^{j \omega})=\mathcal{F}_{d}[x[n]]=\sum_{n=-\infty}^{\infty} x[n] e^{-j \omega n}
  \]
- 逆变换:
  \[
  x[n]=\mathcal{F}_{d}^{-1}[X(e^{j \omega})]=\frac{1}{2 \pi} \int_{-\pi}^{\pi} X(e^{j \omega}) e^{j \omega n} d \omega
  \]

\subsection{主要性质}
- 线性性:\(\mathcal{F}_{d}[a x[n]+b y[n]]=a X(e^{j \omega})+b Y(e^{j \omega})\)
- 时移性:\(\mathcal{F}_{d}[x[n-n_{0}]]=e^{-j \omega n_{0}} X(e^{j \omega})\)
- 频移性:\(\mathcal{F}_{d}[e^{j \omega_{0} n} x[n]]=X(e^{j(\omega-\omega_{0})})\)
- 周期性:\(X(e^{j(\omega+2 \pi)})=X(e^{j \omega})\)
- 共轭对称性:若 \(x[n]\) 实,则 \(X(e^{-j \omega})=X^{*}(e^{j \omega})\)
- 频域微分:\(\mathcal{F}_{d}[n x[n]]=j \frac{d X(e^{j \omega})}{d \omega}\)
- 时域扩展:设 \(x_{k}[n]=\begin{cases}x[n/k] & k|n \\ 0 & \text{其他}\end{cases}\),则 \(X_{k}(e^{j \omega})=X(e^{j k \omega})\)
- 卷积性:\(\mathcal{F}_{d}[x[n] * y[n]]=X(e^{j \omega}) Y(e^{j \omega})\)
- 调制性:\(\mathcal{F}_{d}[x[n] y[n]]=\frac{1}{2 \pi} X(e^{j \omega}) * Y(e^{j \omega})\)

\subsection{奇偶虚实性}
- \(x[n]\) 实偶 \(\Rightarrow X(e^{j \omega})\) 实偶
- \(x[n]\) 实奇 \(\Rightarrow X(e^{j \omega})\) 纯虚奇
- \(x[n]\) 虚偶 \(\Rightarrow X(e^{j \omega})\) 虚偶
- \(x[n]\) 虚奇 \(\Rightarrow X(e^{j \omega})\) 实奇

\subsection{常见DTFT变换对}
\[
\begin{aligned}
\delta[n] &\leftrightarrow 1 \\
\delta[n-n_{0}] &\leftrightarrow e^{-j \omega n_{0}} \\
a^{n} u[n] &\leftrightarrow \frac{1}{1-a e^{-j \omega}} \quad (|a|<1) \\
u[n] &\leftrightarrow \frac{1}{1-e^{-j \omega}}+\pi \sum_{k=-\infty}^{\infty} \delta(\omega-2 \pi k) \\
\text{rect}_{N}[n] &\leftrightarrow \frac{\sin(\omega N/2)}{\sin(\omega/2)} e^{-j \omega(N-1)/2}
\end{aligned}
\]

\section{拉普拉斯变换(双边)}
\subsection{定义}
\[
F(s)=\mathcal{L}[f(t)]=\int_{-\infty}^{\infty} f(t) e^{-s t} dt, \quad s=\sigma+j \omega
\]

\subsection{主要性质}
- 线性性:\(\mathcal{L}[a f(t)+b g(t)]=a F(s)+b G(s)\),ROC至少为 \(R_1 \cap R_2\)
- 时移性:\(\mathcal{L}[f(t-t_{0})]=e^{-s t_{0}} F(s)\)
- 频移性:\(\mathcal{L}[e^{-a t} f(t)]=F(s+a)\),ROC:\(\text{Re}(s+a) \in R\)
- 尺度变换:\(\mathcal{L}[f(a t)]=\frac{1}{|a|} F\left(\frac{s}{a}\right)\),ROC:\(\frac{s}{a} \in R\)
- 时域微分:\(\mathcal{L}[f'(t)]=s F(s)\),ROC至少为 \(R\)
- \(s\) 域微分:\(\mathcal{L}[t f(t)]=-\frac{d F(s)}{d s}\),ROC:\(R\)
- 积分性:\(\mathcal{L}\left[\int_{-\infty}^{t} f(\tau) d \tau\right]=\frac{F(s)}{s}\),ROC至少为 \(R \cap \{\text{Re}(s)>0\}\)
- 卷积性:\(\mathcal{L}[f(t) * g(t)]=F(s) G(s)\),ROC至少为 \(R_1 \cap R_2\)
- 初值定理:\(f(0^+)=\lim_{s \to \infty} s F(s)\)(因果信号)
- 终值定理:\(\lim_{t \to \infty} f(t)=\lim_{s \to 0} s F(s)\)(极点在左半平面或原点)

\subsection{常见拉普拉斯变换对}
\[
\begin{aligned}
\delta(t) &\leftrightarrow 1 \quad (\text{全} \, s \, \text{平面}) \\
u(t) &\leftrightarrow \frac{1}{s} \quad (\text{Re}(s)>0) \\
-u(-t) &\leftrightarrow \frac{1}{s} \quad (\text{Re}(s)<0) \\
e^{-a t} u(t) &\leftrightarrow \frac{1}{s+a} \quad (\text{Re}(s)>-a) \\
-e^{-a t} u(-t) &\leftrightarrow \frac{1}{s+a} \quad (\text{Re}(s)<-a) \\
t e^{-a t} u(t) &\leftrightarrow \frac{1}{(s+a)^2} \quad (\text{Re}(s)>-a) \\
-t e^{-a t} u(-t) &\leftrightarrow \frac{1}{(s+a)^2} \quad (\text{Re}(s)<-a) \\
e^{-a|t|} &\leftrightarrow \frac{2 a}{s^2 - a^2} \quad (-a<\text{Re}(s)<a) \\
t^n u(t) &\leftrightarrow \frac{n!}{s^{n+1}} \quad (\text{Re}(s)>0)
\end{aligned}
\]

\section{z变换}
\subsection{定义}
\[
X(z)=\mathcal{Z}[x[n]]=\sum_{n=-\infty}^{\infty} x[n] z^{-n}
\]

\subsection{主要性质}
- 线性性:\(\mathcal{Z}[a x[n]+b y[n]]=a X(z)+b Y(z)\),ROC至少为 \(R_1 \cap R_2\)
- 时移性:\(\mathcal{Z}[x[n-n_{0}]]=z^{-n_{0}} X(z)\),ROC:\(R\)(可能除去 \(z=0\) 或 \(z=\infty\))
- 尺度变换:\(\mathcal{Z}[a^n x[n]]=X\left(\frac{z}{a}\right)\),ROC:\(|z/a| \in R\)
- \(z\) 域微分:\(\mathcal{Z}[n x[n]]=-z \frac{d X(z)}{d z}\),ROC:\(R\)
- 时域卷积:\(\mathcal{Z}[x[n] * y[n]]=X(z) Y(z)\),ROC至少为 \(R_1 \cap R_2\)
- 差分性质:\(\mathcal{Z}[x[n]-x[n-1]]=(1-z^{-1}) X(z)\),ROC至少为 \(R \cap \{z \neq 0\}\)
- 累加性质:\(\mathcal{Z}\left[\sum_{k=-\infty}^{n} x[k]\right]=\frac{X(z)}{1-z^{-1}}\),ROC至少为 \(R \cap \{|z|>1\}\)
- 初值定理:\(x[0]=\lim_{z \to \infty} X(z)\)(因果序列)
- 终值定理:\(\lim_{n \to \infty} x[n]=\lim_{z \to 1} (z-1) X(z)\)(极点在单位圆内或 \(z=1\))

\subsection{常见z变换对}
\[
\begin{aligned}
\delta[n] &\leftrightarrow 1 \quad (\text{全} \, z \, \text{平面}) \\
u[n] &\leftrightarrow \frac{z}{z-1} \quad (|z|>1) \\
-u[-n-1] &\leftrightarrow \frac{z}{z-1} \quad (|z|<1) \\
a^n u[n] &\leftrightarrow \frac{z}{z-a} \quad (|z|>|a|) \\
-a^n u[-n-1] &\leftrightarrow \frac{z}{z-a} \quad (|z|<<|a|) \\
n a^n u[n] &\leftrightarrow \frac{a z}{(z-a)^2} \quad (|z|>|a|) \\
-n a^n u[-n-1] &\leftrightarrow \frac{a z}{(z-a)^2} \quad (|z|<<|a|) \\
\cos(\omega_{0} n) u[n] &\leftrightarrow \frac{z(z-\cos\omega_{0})}{z^2 - 2 z \cos\omega_{0} + 1} \\
\sin(\omega_{0} n) u[n] &\leftrightarrow \frac{z \sin\omega_{0}}{z^2 - 2 z \cos\omega_{0} + 1}
\end{aligned}
\]

\section{连续时间LTI系统分析}
\subsection{系统函数}
\[
H(s)=\frac{Y(s)}{X(s)}=\mathcal{L}[h(t)]
\]

\subsection{稳定性判据}
- BIBO稳定:所有极点在左半平面(\(\text{Re}(p_j)<0\))
- 临界稳定:极点在虚轴上
- 不稳定:至少一个极点在右半平面(\(\text{Re}(p_j)>0\))

\subsection{频率响应}
\[
H(j\omega)=H(s)\bigg|_{s=j\omega}
\]
- 幅频特性:\(|H(j\omega)|\)
- 相频特性:\(\angle H(j\omega)\)

\subsection{无失真传输条件}
\[
y(t)=K x(t-t_d)
\]
- 幅频条件:\(|H(j\omega)|=K\)(常数)
- 相频条件:\(\angle H(j\omega)=-\omega t_d\)
- 理想频率响应:\(H(j\omega)=K e^{-j\omega t_d}\)

\subsection{群时延与相时延}
- 群时延(包络时延):\(\tau_g(\omega)=-\frac{d \theta(\omega)}{d \omega}\)(\(\theta(\omega)=\angle H(j\omega)\))
- 相时延(载波时延):\(\tau_p(\omega)=-\frac{\theta(\omega)}{\omega}\)
- 线性相位时:\(\tau_g=\tau_p=t_d\)

\subsection{部分分式展开}
对于 \(H(s)=\frac{N(s)}{D(s)}\),若极点 \(p_i\) 为单极点:
\[
H(s)=\sum_{i} \frac{r_i}{s-p_i}, \quad r_i=\left[(s-p_i) H(s)\right]_{s=p_i}
\]

\section{离散时间LTI系统分析}
\subsection{系统函数}
\[
H(z)=\frac{Y(z)}{X(z)}=\mathcal{Z}[h[n]]
\]

\subsection{稳定性判据}
- BIBO稳定:所有极点在单位圆内(\(|p_j|<1\))
- 临界稳定:极点在单位圆上
- 不稳定:至少一个极点在单位圆外(\(|p_j|>1\))

\subsection{频率响应}
\[
H(e^{j\omega})=H(z)\bigg|_{z=e^{j\omega}}
\]

\subsection{差分方程与系统函数关系}
差分方程:
\[
\sum_{k=0}^{N} a_k y[n-k]=\sum_{k=0}^{M} b_k x[n-k]
\]
系统函数:
\[
H(z)=\frac{Y(z)}{X(z)}=\frac{\sum_{k=0}^{M} b_k z^{-k}}{\sum_{k=0}^{N} a_k z^{-k}}
\]

\section{因果性与稳定性}
\subsection{连续系统(s域)}
- 因果性:\(H(s)\) 是真有理函数(分子次数≤分母次数),ROC为 \(\text{Re}(s)>\sigma_0\)
- 稳定性:ROC包含虚轴,所有极点 \(\text{Re}(p)<0\)
- 因果稳定:ROC为 \(\text{Re}(s)>\sigma_0\) 且 \(\sigma_0<0\)

\subsection{离散系统(z域)}
- 因果性:ROC为 \(|z|>r\)(某圆外)
- 稳定性:ROC包含单位圆(\(|z|=1\)),所有极点 \(|p|<1\)
- 因果稳定:ROC为 \(|z|>r\) 且 \(r<1\)

\section{系统连接}
- 串联:\(H(s)=H_1(s)H_2(s)\) 或 \(H(z)=H_1(z)H_2(z)\)
- 并联:\(H(s)=H_1(s)+H_2(s)\) 或 \(H(z)=H_1(z)+H_2(z)\)
- 反馈:\(H(s)=\frac{H_1(s)}{1 \pm H_1(s)H_2(s)}\) 或 \(H(z)=\frac{H_1(z)}{1 \pm H_1(z)H_2(z)}\)(+负反馈,-正反馈)

\section{拉普拉斯与z变换类比}
\subsection{稳定性对应}
| 拉普拉斯变换(连续) | z变换(离散) |
|----------------------|---------------|
| 稳定:\(\text{Re}(s)<0\) | 稳定:\(|z|<1\) |
| 临界稳定:虚轴 | 临界稳定:单位圆 |
| 不稳定:\(\text{Re}(s)>0\) | 不稳定:\(|z|>1\) |

\subsection{映射关系}
\[
z=e^{sT} \quad \text{或} \quad s=\frac{1}{T} \ln z
\]
- s左半平面 ↔ z单位圆内
- s虚轴 ↔ z单位圆
- s右半平面 ↔ z单位圆外
- 主频带:\(-\pi/T < \omega < \pi/T\)

\section{采样理论}
\subsection{冲激串采样}
- 采样信号:\(x_p(t)=x(t) \sum_{n=-\infty}^{\infty} \delta(t-nT)\)
- 频域特性:
  \[
  X_p(j\omega)=\frac{1}{T} \sum_{k=-\infty}^{\infty} X(j(\omega-k\omega_s)), \quad \omega_s=\frac{2\pi}{T}
  \]
  (频谱周期延拓,周期为 \(\omega_s\))

\subsection{奈奎斯特采样定理}
- 条件:\(x(t)\) 带限于 \(\omega_M\)(\(X(j\omega)=0, |\omega|>\omega_M\))
- 采样频率要求:\(\omega_s \geq 2\omega_M\)(奈奎斯特率),可无失真恢复

\subsection{信号恢复}
- 理想低通滤波器:
  \[
  H_r(j\omega)=\begin{cases}T & |\omega|<\omega_c \\ 0 & |\omega| \geq \omega_c\end{cases}, \quad \omega_M<\omega_c<\omega_s-\omega_M
  \]
- 零阶保持(ZOH):
  \[
  x_0(t)=\sum_{n=-\infty}^{\infty} x(nT) \text{rect}\left(\frac{t-nT-T/2}{T}\right)
  \]
  频域:\(X_0(j\omega)=H_0(j\omega)X_p(j\omega)\),其中 \(H_0(j\omega)=T \text{sinc}\left(\frac{\omega T}{2}\right) e^{-j\omega T/2}\)
- 线性内插:
  \[
  x_1(t)=\sum_{n=-\infty}^{\infty} x(nT) \text{tri}\left(\frac{t-nT}{T}\right)
  \]
  频域:\(X_1(j\omega)=H_1(j\omega)X_p(j\omega)\),其中 \(H_1(j\omega)=T \text{sinc}^2\left(\frac{\omega T}{2}\right)\)

\subsection{离散时间处理连续信号流程}
1. C/D转换:\(x[n]=x_c(nT)\),频域关系:
   \[
   X(e^{j\omega})=\frac{1}{T} \sum_{k=-\infty}^{\infty} X_c\left(j\frac{\omega-2\pi k}{T}\right)
   \]
2. 离散处理:\(Y(e^{j\omega})=H(e^{j\omega})X(e^{j\omega})\),等效连续频率 \(\Omega=\omega T\)
3. D/C转换:
   - 理想重建:\(y_c(t)=\sum_{n=-\infty}^{\infty} y[n] \frac{\sin(\pi(t-nT)/T)}{\pi(t-nT)/T}\)
   - 零阶保持:\(y_c(t)=\sum_{n=-\infty}^{\infty} y[n] \text{rect}\left(\frac{t-nT-T/2}{T}\right)\)
4. 等效连续系统:\(H_{eff}(j\Omega)=H(e^{j\Omega/T})\)(\(|\Omega|<\pi/T\))

\subsection{混叠现象}
- 产生:\(\omega_s<2\omega_M\) 时,频谱重叠,高频成分"伪装"成低频
- 解决:预滤波(抗混叠滤波器)

\subsection{注意事项}
1. 审题明确求解目标(单位冲激响应/单位阶跃响应等)
2. 注意隐含的因果性和稳定性条件
3. 拉普拉斯变换与z变换需明确收敛域

\end{multicols*}
\end{document}
