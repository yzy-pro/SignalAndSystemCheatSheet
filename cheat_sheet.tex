\documentclass{article}
\usepackage{amsmath,amssymb}
\allowdisplaybreaks
\usepackage{multicol}
\usepackage{geometry}
\usepackage{ctex}
\usepackage{graphicx}
\usepackage{listings}
\usepackage{xcolor}
\usepackage{fancyhdr}
\usepackage{titlesec}
\titlespacing*{\section}{0pt}{0pt}{0pt}
\titlespacing*{\subsection}{0pt}{-4pt}{-12pt}
\titleformat*{\section}{\small\bfseries}
\titleformat*{\subsection}{\footnotesize\bfseries}
\AtBeginDocument{
    \small
    \setlength{\abovedisplayskip}{0pt}
    \setlength{\belowdisplayskip}{0pt}
    \setlength{\abovedisplayshortskip}{0pt}
    \setlength{\belowdisplayshortskip}{0pt}
}
\usepackage{hyperref}
\usepackage{tikz}
\geometry{a4paper, landscape, margin=1cm}

\title{信号处理核心公式汇总}
\author{}
\date{}

\begin{document}
\begin{multicols*}{5}

\section{卷积与奇异函数}
\subsection{定义}
\begin{align*}
&x(t) * h(t)=\int_{-\infty}^{\infty} x(\tau) h(t-\tau) d \tau \\
&x[n] * h[n]=\sum_{k=-\infty}^{\infty} x[k] h[n-k] 
\end{align*}
\subsection{重要性质}
\begin{align*}
  &1. \delta(at) = \frac{1}{|a|} \delta(t) \\
&1. \text{若 } x_1(t) * x_2(t) = y(t) \text{,则 } \\
&x_1(t - t_1) * x_2(t - t_2) = y(t - t_1 - t_2) \\
&2. E \text{rect}\left(\frac{t}{\tau}\right)\text{自卷积}= E^2 \tau \text{tri}\left(\frac{t}{2\tau}\right)
\\
&3. \delta(t-t_{0}) * x(t)=x(t-t_{0}) \\
&4. u(t-t_{0})* x(t)=\int_{-\infty}^{t-t_{0}} x(\tau) d \tau
\\
&5. \delta'(t) x(t)= x(0)\delta'(t) - x'(0)\delta(t) \\
&6. \delta'(t-t_{0}) * x(t)= - x'(t_{0})
\end{align*}

\section{傅里叶级数}
连续周期信号:$f(t)=f(t+T)$,基波角频率 $\omega_{0}=\frac{2 \pi}{T}$
\subsection{定义}
\begin{align*}
&f(t)=\sum_{n=-\infty}^{\infty} c_{n} e^{j n \omega_{0} t} \\
&\text{其中傅里叶系数:} \\
&c_{n}=\frac{1}{T} \int_{T} f(t) e^{-j n \omega_{0} t} dt
\end{align*}

\subsection{核心关系与性质}
\begin{align*}
&\text{1. 系数关系(}f(t)\text{ 为实信号):} \\
&c_{0}=a_{0}, \quad c_{n}=\frac{a_{n}-j b_{n}}{2}, \quad c_{-n}=c_{n}^{*} \\
&\text{2. 线性性:}a f(t)+b g(t) \leftrightarrow a c_{n}+b d_{n} \\
&\text{3. 时移性:}f(t-t_{0}) \leftrightarrow c_{n} e^{-j n \omega_{0} t_{0}} \\
&\text{4. 共轭对称性:}f(t) \text{ 实 } \Rightarrow c_{-n}=c_{n}^{*} \\
&\text{5. 帕塞瓦尔定理:} \\
&\frac{1}{T} \int_{T}|f(t)|^{2} dt=\sum_{n=-\infty}^{\infty}\left|c_{n}\right|^{2}
\end{align*}

\section{傅里叶变换}
\subsection{定义}
\begin{align*}
  &F(\omega)=\int_{-\infty}^{\infty} f(t) e^{-j \omega t} dt
  \\
  &f(t)=\frac{1}{2 \pi} \int_{-\infty}^{\infty} F(\omega) e^{j \omega t} d \omega
\end{align*}
\subsection{重要性质}
\begin{align*}
&\text{1.线性:}af(t)+bg(t) 
\\
&\rightarrow aF(j\omega)+bG(j\omega)
\\
&\text{2.时移:}f(t-t_{0}) \rightarrow F(j\omega) e^{-j \omega t_{0}}
\\
&\text{3.频移:}f(t) e^{j \omega_{0} t} \rightarrow F(j(\omega-\omega_{0}))
\\
&\text{4.频域扩展:}f(a t) \rightarrow \frac{1}{|a|} F\left(\frac{j\omega}{a}\right)
\\
&\text{5.对偶性:}F(t) \rightarrow 2 \pi f(-j\omega)
\\
&\text{6. 时域微分:}\frac{d^{n} f(t)}{d t^{n}} \rightarrow (j\omega)^{n}F(j\omega)
\\
&\text{7. 频域微分:}t^{n} f(t) \rightarrow j^{n} \frac{d^{n} F(j\omega)}{d \omega^{n}}
\\
&\text{8. 时域积分:}
\\
&\int_{-\infty}^{t} f(\tau) d \tau \rightarrow \frac{F(j\omega)}{j \omega}+\pi F(0) \delta(\omega)
\\
&\text{9. 时域卷积:}
\\
&f(t) * g(t) \rightarrow F(j\omega) G(j\omega)
\\
&\text{10. 频域卷积:}
\\
&f(t) g(t) \rightarrow \frac{1}{2 \pi} F(j\omega) * G(j\omega)
\\
&\text{11. 帕塞瓦尔定理:}
\\
&\int_{-\infty}^{\infty}|f(t)|^{2} dt=\frac{1}{2 \pi} \int_{-\infty}^{\infty}|F(j\omega)|^{2} d \omega
\\
&\text{12. 共轭对称性:} 
\\
&f(t) = \text{Re}\{f(t)\} \Rightarrow F(-j\omega)=F^{*}(j\omega)
\\
&f(t)\text{实偶} \Rightarrow F(j\omega)\text{实偶}
\\
&f(t)\text{实奇} \Rightarrow F(j\omega)\text{纯虚奇}
\\
&f(t)\text{虚偶} \Rightarrow F(j\omega)\text{虚偶}
\\
&f(t)\text{虚奇} \Rightarrow F(j\omega)\text{实奇}
\end{align*}
\subsection{奇偶分解}
\begin{align*}
&f_{e}(t)=\frac{f(t)+f(-t)}{2} \Leftrightarrow 
\\
&F_{e}(\omega)=\text{Re}\{F(\omega)\}
\\
&f_{o}(t)=\frac{f(t)-f(-t)}{2} \Leftrightarrow 
\\
&F_{o}(\omega)=j \text{Im}\{F(\omega)\}
\\
&f(t)=f_{e}(t)+f_{o}(t),
\\
&F(\omega)=F_{e}(\omega)+F_{o}(\omega)
\end{align*}
\subsection{常见傅里叶变换对}
\begin{align*}
&\delta(t) \leftrightarrow 1 
\\
&1 \leftrightarrow 2 \pi \delta(\omega) 
\\
&e^{j \omega_{0} t} \leftrightarrow 2 \pi \delta(\omega-\omega_{0}) \\
&\cos(\omega_{0} t) \leftrightarrow \pi\left[\delta(\omega-\omega_{0})+\delta(\omega+\omega_{0})\right] 
\\
&\sin(\omega_{0} t) \leftrightarrow \frac{\pi}{j}\left[\delta(\omega-\omega_{0})-\delta(\omega+\omega_{0})\right] 
\\
&e^{-a t} u(t) \leftrightarrow \frac{1}{a+j \omega} \quad (Re(a)>0) 
\\
&t e^{-a t} u(t) \leftrightarrow \frac{1}{(a+j \omega)^{2}} \quad (Re(a)>0) 
\\
&e^{-a|t|} \leftrightarrow \frac{2 a}{a^{2}+\omega^{2}} \quad (Re(a)>0) 
\\
&u(t) \leftrightarrow \pi \delta(\omega)+\frac{1}{j \omega} 
\\
&\text{rect}\left(\frac{t}{\tau}\right) \leftrightarrow \tau \cdot \frac{\sin(\omega \tau/2)}{\omega \tau/2} 
\\
&\text{sinc}(t)=\frac{\sin(\pi t)}{\pi t} \leftrightarrow \text{rect}\left(\frac{\omega}{2 \pi}\right) 
\\
&\text{tri}\left(\frac{t}{\tau}\right) \leftrightarrow \tau \cdot \frac{\sin^{2}(\omega \tau/2)}{(\omega \tau/2)^{2}} 
\\
&\sum_{n=-\infty}^{\infty} \delta(t-n T) \leftrightarrow \frac{2 \pi}{T} \sum_{k=-\infty}^{\infty} \delta(\omega-k \omega_{0})
\end{align*}
\subsection{周期信号的傅里叶变换}
\begin{align*}
&\text{若}f(t)=\sum_{n=-\infty}^{\infty} {c_n} e^{j n \omega_{0} t}
\\
&\text{则}F(\omega)=2 \pi \sum_{n=-\infty}^{\infty} c_{n} \delta(\omega-n \omega_{0})
\\
&\text{若}f(t)=\sum_{k=-\infty}^{\infty} f_{0}(t-k T)
\\
&\text{则}
  F(\omega)=F_{0}(\omega) \cdot \frac{2 \pi}{T} \sum_{n=-\infty}^{\infty} \delta(\omega-n \omega_{0})
\\
&\text{其中}c_{n}=\frac{1}{T} F_{0}(n \omega_{0})
\end{align*}

\section{离散时间傅里叶变换(DTFT)}
\subsection{定义}
\begin{align*}
&X(e^{j \omega})=\sum_{n=-\infty}^{\infty} x[n] e^{-j \omega n} \\
&x[n]=\frac{1}{2 \pi} \int_{-\pi}^{\pi} X(e^{j \omega}) e^{j \omega n} d \omega
\end{align*}

\subsection{主要性质}
\begin{align*}
&\text{1. 线性:}a x[n]+b y[n] \rightarrow a X(e^{j \omega})+b Y(e^{j \omega}) \\
&\text{2. 时移:}x[n-n_{0}] \rightarrow e^{-j \omega n_{0}} X(e^{j \omega}) \\
&\text{3. 频移:}e^{j \omega_{0} n} x[n] \rightarrow X(e^{j(\omega-\omega_{0})}) \\
&\text{4. 周期性:}X(e^{j(\omega+2 \pi)}) \rightarrow X(e^{j \omega}) \\
&\text{5. 共轭对称:若 }x[n]\text{ 实}
\\
&\text{,则 }X(e^{-j \omega}) \rightarrow X^{*}(e^{j \omega}) \\
&\text{6. 频域微分:}n x[n] \rightarrow  \frac{d X(e^{j \omega})}{d \omega} \\
&\text{7. 时域扩展:}
\\
&x_{k}[n] \rightarrow \begin{cases}x[n/k] & k|n \\ 0 & \text{otherwise}\end{cases}
\\
&\text{则}X_{k}(e^{j \omega}) \rightarrow X(e^{j k \omega}) \\
&\text{8. 卷积:}x[n] * y[n] \rightarrow X(e^{j \omega}) Y(e^{j \omega}) \\
&\text{9. 调制:}x[n] y[n] \rightarrow \frac{1}{2 \pi} X(e^{j \omega}) * Y(e^{j \omega})
\end{align*}

\subsection{常见DTFT变换对}
\begin{align*}
&\delta[n] \leftrightarrow 1 \\
&\delta[n-n_{0}] \leftrightarrow e^{-j \omega n_{0}} \\
&a^{n} u[n] \leftrightarrow \frac{1}{1-a e^{-j \omega}} \quad (|a|<1) \\
&u[n] \leftrightarrow \frac{1}{1-e^{-j \omega}}+\pi \sum_{k=-\infty}^{\infty} \delta(\omega-2 \pi k) \\
&\text{rect}_{N}[n] \leftrightarrow \frac{\sin(\omega N/2)}{\sin(\omega/2)} e^{-j \omega(N-1)/2}
\end{align*}

\section{拉普拉斯变换(双边)}
\subsection{定义}
\begin{align*}
&F(s)=\int_{-\infty}^{\infty} f(t) e^{-s t} dt, \quad s=\sigma+j \omega
\end{align*}

\subsection{主要性质}
\begin{align*}
&\text{1. 线性:}a f(t)+b g(t) \rightarrow 
\\
&a F(s)+b G(s)\text{,至少}R_1 \cap R_2 \\
&\text{2. 时移:}f(t-t_{0}) \rightarrow e^{-s t_{0}} F(s)\text{,R} \\
&\text{3. 频移:}e^{-a t} f(t) \rightarrow F(s+a)
\\
&\text{,}\text{Re}(s+a) \in R \\
&\text{4. 尺度变换:}f(a t) \rightarrow \frac{1}{|a|} F\left(\frac{s}{a}\right)
\text{,}\frac{s}{a} \in R \\
&\text{5. 时域微分:}f'(t) \rightarrow s F(s)\text{,至少}R \\
&\text{6. }s\text{ 域微分:}t f(t) \rightarrow -\frac{d F(s)}{d s}\text{,}R \\
&\text{7. 积分:}\int_{-\infty}^{t} f(\tau) d \tau \rightarrow \frac{F(s)}{s}
\\
&\text{,至少}R \cap \{\text{Re}(s)>0\} \\
&\text{8. 卷积:}f(t) * g(t) \rightarrow F(s) G(s)\text{至少}R_1 \cap R_2 \\
&\text{9. 初值:}f(0^+) = \lim_{s \to \infty} s F(s)\text{(因果信号)} \\
&\text{10. 终值:}\lim_{t \to \infty} f(t) = \lim_{s \to 0} s F(s)
\\
&\text{(极点在左半平面或原点)}
\end{align*}

\subsection{常见拉普拉斯变换对}
\begin{align*}
&\delta(t) \leftrightarrow 1 \quad (\text{全} \, s \, \text{平面}) \\
&u(t) \leftrightarrow \frac{1}{s} \quad (\text{Re}(s)>0) \\
&-u(-t) \leftrightarrow \frac{1}{s} \quad (\text{Re}(s)<0) \\
&e^{-a t} u(t) \leftrightarrow \frac{1}{s+a} \quad (\text{Re}(s)>-a) \\
&-e^{-a t} u(-t) \leftrightarrow \frac{1}{s+a} \quad (\text{Re}(s)<-a) \\
&t e^{-a t} u(t) \leftrightarrow \frac{1}{(s+a)^2} \quad (\text{Re}(s)>-a) \\
&-t e^{-a t} u(-t) \leftrightarrow \frac{1}{(s+a)^2} \quad (\text{Re}(s)<-a) \\
&e^{-a|t|} \leftrightarrow \frac{2 a}{s^2 - a^2} \quad (-a<\text{Re}(s)<a) \\
&t^n u(t) \leftrightarrow \frac{n!}{s^{n+1}} \quad (\text{Re}(s)>0)
\end{align*}

\section{z变换}
\subsection{定义}
\begin{align*}
&X(z)=\mathcal{Z}[x[n]]=\sum_{n=-\infty}^{\infty} x[n] z^{-n}
\end{align*}
\subsection{主要性质}
\begin{align*}
&\text{1.线性:}a x[n]+b y[n]=
\\
&a X(z)+b Y(z)\text{,至少}R_1 \cap R_2 
\\
&\text{2. 时移:}x[n-n_{0}]=z^{-n_{0}} X(z)
\\
&\text{,}R\text{(可能除去 }z=0\text{ 或 }z=\infty\text{)} 
\\
&\text{3.尺度变换:}a^n x[n]=X\left(\frac{z}{a}\right)\text{,}|z/a| \in R \\
&\text{4. }\text{z域微分:}n x[n]=-z \frac{d X(z)}{d z}\text{,}R \\
&\text{5.时域卷积:}x[n] * y[n]=X(z) Y(z)
\\
&\text{,至少}R_1 \cap R_2 \\
&\text{6. 差分性质:}x[n]-x[n-1]=
\\
&(1-z^{-1}) X(z)\text{,至少}R \cap \{z \neq 0\} \\
&\text{7. 累加性质:}\sum_{k=-\infty}^{n} x[k]=\frac{X(z)}{1-z^{-1}}
\\
&\text{,至少}R \cap \{|z|>1\} \\
&\text{8. 初值:}x[0]=\lim_{z \to \infty} X(z)\text{(因果序列)} \\
&\text{9. 终值:}\lim_{n \to \infty} x[n]=\lim_{z \to 1} (z-1) X(z)
\\
&\text{(极点在单位圆内或 }z=1\text{)}
\\
&\text{10. 实信号的s变换的零极点共轭出现}
\end{align*}

\subsection{常见z变换对}
\begin{align*}
&\delta[n] \leftrightarrow 1 \text{,全平面} \\
&u[n] \leftrightarrow \frac{z}{z-1}\text{,}|z|>1 \\
&-u[-n-1] \leftrightarrow \frac{z}{z-1}\text{,}|z|<1 \\
&a^n u[n] \leftrightarrow \frac{z}{z-a} (|z|>|a|) \\
&-a^n u[-n-1] \leftrightarrow \frac{z}{z-a} \text{,}|z|<|a| \\
&n a^n u[n] \leftrightarrow \frac{a z}{(z-a)^2} \text{,}|z|>|a| \\
&-n a^n u[-n-1] \leftrightarrow \frac{a z}{(z-a)^2} \text{,}|z|<|a| \\
&\cos(\omega_{0} n) u[n] \leftrightarrow \frac{z(z-\cos\omega_{0})}{z^2 - 2 z \cos\omega_{0} + 1} \\
&\sin(\omega_{0} n) u[n] \leftrightarrow \frac{z \sin\omega_{0}}{z^2 - 2 z \cos\omega_{0} + 1}
\end{align*}

\section{连续时间LTI系统分析}
\subsection{系统函数}
\begin{align*}
&H(s)=\frac{Y(s)}{X(s)}=\mathcal{L}[h(t)]
\end{align*}

\subsection{稳定性判据}
\begin{align*}
&\text{1. 稳定:所有极点在左半平面} \\
&\text{2. 临界稳定:极点在虚轴上} \\
&\text{3. 不稳定:至少一个极点在右半平面}
\end{align*}

\subsection{频率响应}
\begin{align*}
&H(j\omega)=H(s)\bigg|_{s=j\omega} \\
&\text{幅频特性:}|H(j\omega)| \\
&\text{相频特性:}\angle H(j\omega)
\end{align*}

\subsection{无失真传输条件}
\begin{align*}
&\text{时域}y(t)=K x(t-t_d) \\
&\text{频域}H(j\omega)=K e^{-j\omega t_d}
\end{align*}

\subsection{群时延与相时延}
\begin{align*}
&\text{群时延(包络时延):}\tau_g(\omega)=-\frac{d \theta(\omega)}{d \omega}
\\
&\text{(}\theta(\omega)=\angle H(j\omega)\text{)} \\
&\text{相时延(载波时延):}\tau_p(\omega)=-\frac{\theta(\omega)}{\omega} \\
&\text{线性相位时:}\tau_g=\tau_p=t_d
\end{align*}

\subsection{部分分式展开}
\begin{align*}
  &\text{
对于 $H(s)=\frac{N(s)}{D(s)}$,若极点 $p_i$ 为单极点:}
\\
&H(s)=\sum_{i} \frac{r_i}{s-p_i}, \\
& r_i=\left[(s-p_i) H(s)\right]_{s=p_i}
\end{align*}

\section{离散时间LTI系统分析}
\subsection{系统函数}
\begin{align*}
&H(z)=\frac{Y(z)}{X(z)}=\mathcal{Z}[h[n]]
\end{align*}

\subsection{稳定性判据}
\begin{align*}
&\text{1. 稳定:所有极点在单位圆内} \\
&\text{2. 临界稳定:极点在单位圆上} \\
&\text{3. 不稳定:至少一个极点在单位圆外}
\end{align*}

\subsection{频率响应}
\begin{align*}
&H(e^{j\omega})=H(z)\bigg|_{z=e^{j\omega}}
\end{align*}

\subsection{差分方程与系统函数关系}
\begin{align*}
&\text{差分方程:} \\
&\sum_{k=0}^{N} a_k y[n-k]=\sum_{k=0}^{M} b_k x[n-k] \\
&\text{系统函数:} \\
&H(z)=\frac{Y(z)}{X(z)}=\frac{\sum_{k=0}^{M} b_k z^{-k}}{\sum_{k=0}^{N} a_k z^{-k}}
\end{align*}

\section{因果性与稳定性}
\subsection{连续系统(s域)}
\begin{align*}
&\text{1. 因果性:}H(s)
\text{ 是真有理函数}
\\
&\text{(分子次数}\le\text{分母次数),ROC为 }
\\
&\text{Re}(s)>\sigma_0 \\
&\text{2. 稳定性:ROC包含虚轴} \\
&\text{3. 因果稳定:ROC为 }\text{Re}(s)>\sigma_0
\\
&\text{ 且 }\sigma_0<0
\text{,所有极点 }\text{Re}(p)<0
\end{align*}

\subsection{离散系统(z域)}
\begin{align*}
&\text{1. 因果性:ROC为 }|z|>r\text{(某圆外)} \\
&\text{2. 稳定性:ROC包含单位圆(}|z|=1\text{)} \\
&\text{3. 因果稳定:ROC为 }|z|>r\text{ 且 }r<1
\\
&\text{所有极点}|p|<1
\end{align*}

\section{系统连接}
\begin{align*}
&\text{1. 串联:}H(s)=H_1(s)H_2(s) 
\\
&\text{2. 并联:}H(s)=H_1(s)+H_2(s)
\\
&\text{3. 反馈:}H(s)=\frac{H_1(s)}{1 \pm H_1(s)H_2(s)}
\\
&\text{(+负反馈,-正反馈)}
\end{align*}

\section{拉普拉斯与z变换类比}
\subsection{稳定性对应}
\begin{align*}
&\text{s变换(连续)}  & \text{z变换(离散)} \\
&\text{稳定:}\text{Re}(s)<0  & \text{}|z|<1 \\
&\text{临界稳定:虚轴}  & \text{单位圆} \\
&\text{不稳定:}\text{Re}(s)>0  & \text{}|z|>1
\end{align*}

\subsection{映射关系}
\begin{align*}
&z=e^{sT} \quad \text{或} \quad s=\frac{1}{T} \ln z \\
&s\text{左半平面} \leftrightarrow z\text{单位圆内} \\
&s\text{虚轴} \leftrightarrow z\text{单位圆} \\
&s\text{右半平面} \leftrightarrow z\text{单位圆外} \\
&\text{主频带:}-\pi/T < \omega < \pi/T
\end{align*}

\section{采样理论}
\subsection{冲激串采样}
\begin{align*}
&\text{采样信号:}x_p(t)=x(t) \sum_{n=-\infty}^{\infty} \delta(t-nT) \\
&X_p(j\omega)=\frac{1}{T} \sum_{k=-\infty}^{\infty} X(j(\omega-k\omega_s)), 
\\
& \omega_s=\frac{2\pi}{T} \\
&\text{(频谱周期延拓,周期为 }\omega_s\text{)}
\end{align*}

\subsection{奈奎斯特采样定理}
\begin{align*}
&x(t)\text{ 带限于 }\omega_M\text{(}X(j\omega)=0, |\omega|>\omega_M\text{)} \\
&\text{则采样频率要求:}\omega_s \geq 2\omega_M\text{(奈奎斯特率)}
\end{align*}

\subsection{信号恢复}
\begin{align*}
&\text{1. 理想低通滤波器:} \\
&H_r(j\omega)=\begin{cases}T & |\omega|<\omega_c \\ 0 & |\omega| \geq \omega_c\end{cases}, \quad \omega_M<\omega_c<\omega_s-\omega_M \\
&\text{2. 零阶保持(ZOH):} \\
&x_0(t)=\sum_{n=-\infty}^{\infty} x(nT) \text{rect}\left(\frac{t-nT-T/2}{T}\right) \\
&\text{频域:}X_0(j\omega)=H_0(j\omega)X_p(j\omega)\text{,其中 }H_0(j\omega)=T \text{sinc}\left(\frac{\omega T}{2}\right) e^{-j\omega T/2} \\
&\text{3. 线性内插:} \\
&x_1(t)=\sum_{n=-\infty}^{\infty} x(nT) \text{tri}\left(\frac{t-nT}{T}\right) \\
&\text{频域:}X_1(j\omega)=H_1(j\omega)X_p(j\omega)\text{,其中 }H_1(j\omega)=T \text{sinc}^2\left(\frac{\omega T}{2}\right)
\end{align*}

\subsection{离散时间处理连续信号流程}
\begin{align*}
&\text{1. C/D转换:}x[n]=x_c(nT)\text{,频域关系:} \\
&X(e^{j\omega})=\frac{1}{T} \sum_{k=-\infty}^{\infty} X_c\left(j\frac{\omega-2\pi k}{T}\right) \\
&\text{2. 离散处理:}Y(e^{j\omega})=H(e^{j\omega})X(e^{j\omega})\text{,等效连续频率 }\Omega=\omega T \\
&\text{3. D/C转换:} \\
&\quad \text{理想重建:}y_c(t)=\sum_{n=-\infty}^{\infty} y[n] \frac{\sin(\pi(t-nT)/T)}{\pi(t-nT)/T} \\
&\quad \text{零阶保持:}y_c(t)=\sum_{n=-\infty}^{\infty} y[n] \text{rect}\left(\frac{t-nT-T/2}{T}\right) \\
&\text{4. 等效连续系统:}H_{eff}(j\Omega)=
\\
&H(e^{j\Omega/T})\text{(}|\Omega|<\pi/T\text{)}
\end{align*}

\subsection{混叠现象}
\begin{align*}
&\text{产生:}\omega_s<2\omega_M\text{ 时,频谱重叠,高频成分"伪装"成低频} \\
&\text{解决:预滤波(抗混叠滤波器)}
\end{align*}

\section{注意事项}
\begin{align*}
&\text{1. 看清题目(单位冲激/阶跃响应等)} \\
&\text{2. 注意隐含的因果性和稳定性条件} \\
&\text{3. 拉普拉斯变换与z变换需明确收敛域}
\end{align*}

\end{multicols*}
\end{document}
